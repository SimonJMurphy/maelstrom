\voffset -10mm
\documentclass[a4paper,fleqn,usenatbib]{mnras}
\usepackage{newtxtext,newtxmath}
% Depending on your LaTeX fonts installation, you might get better results with one of these:
%\usepackage{mathptmx}
%\usepackage{txfonts}
\usepackage[T1]{fontenc}
\usepackage{ae,aecompl}
\usepackage{graphicx}	% Including figure files
\usepackage{amsmath}	% Advanced maths commands
\usepackage{amssymb}	% Extra maths symbols
\usepackage{booktabs}	% Pretty tables for all!
\usepackage{url}
\newcommand{\blu}{\textcolor{blue} }
\newcommand{\red}{\textcolor{red} }
\newcommand{\specline}[3]{{#1}\,{\sc #2}\:{#3}}


\title[Maelstrom]{Stellar Clocks -- Introducing Maelstrom}

\author[DFM and SJM]{
Daniel Foreman-Mackey,$^{1}$\thanks{E-mail: foreman.mackey@gmail.com (DFM)}
Simon J. Murphy,$^{2\thanks{E-mail: simon.murphy@sydney.edu.au (SJM)}}$
David W. Hogg,$^{3\thanks{E-mail: david.hogg@nyu.edu (DWH)}}$
\\
% List of institutions
$^{1}$ Flatiron Institute, NYC, USA.\\
$^{2}$ Sydney Institute for Astronomy (SIfA), School of Physics, University of Sydney, NSW 2006, Australia\\
$^{1}$ NYU, NYC, USA.\\
}

% These dates will be filled out by the publisher
\date{Accepted XXX. Received YYY; in original form ZZZ}
% Enter the current year, for the copyright statements etc.
\pubyear{2015}
% Don't change these lines
\begin{document}
\label{firstpage}
\pagerange{\pageref{firstpage}--\pageref{lastpage}}
\maketitle


\begin{abstract}
Just some notes.
\end{abstract}


%\begin{keywords}
%keyword1 -- keyword2 -- keyword3
%\end{keywords}

%%%%%%%%%%%%%%%%%%%%%%%%%%%%%%%%%%%%%%%%%%%%%

%%%%%%%%%%%%%%%%% BODY OF PAPER %%%%%%%%%%%%%%%%%%

\section{Introduction}

Pulsating stars with coherent modes can be used as clocks. In a binary system, the pulsating star's orbit around the barycentre leads to a change in path length for starlight travelling to Earth. The time-dependent luminosity variations due to pulsation therefore include a time-delay term, $\tau$:
\begin{eqnarray}
y(t) = \sum_{j=1}^J \left[ A_j \cos (\omega_j[t-\tau]) + B_j \sin(\omega_j[t-\tau]) \right] + \epsilon
\label{eq:luminosity}
\end{eqnarray}
where $A_j$ is the amplitude and $\omega_j = 2\uppi\nu_j$ is the angular frequency of mode $j$, and $\epsilon$ describes variance unaccounted for by our pulsation model of $J$ modes. We assume that $\epsilon$ is Gaussian distributed with variance $\sigma^2$ (encapsulating both measurement uncertainty and model mis-specification). Note that $\tau$ is defined to be positive when the star is on the far side of the barycentre. Following \citet{murphyetal2016b}, the time delay can be expressed as a function of the true anomaly, $f$,
\begin{eqnarray}
\tau = - \frac{a_1 \sin i}{\rm c} \frac{1-e^2}{1+e \cos f} \sin(f+\varpi)
\label{eq:delay}
\end{eqnarray}
in units of seconds, corresponding to a change in path length measured in light-seconds. We follow the previous convention that $a_1 \sin i$ denotes the projected semimajor axis, $e$ is the eccentricity, $\varpi$ is the angle between the node and the periastron, and c is the speed of light.

In previous papers in this series \citep{murphyetal2014, murphy&shibahashi2015, murphyetal2016b}, time-delay curves were extracted for pulsating binaries (PBs) by segmenting the \textit{Kepler} light curve. The measured time delays were the average over the corresponding segments, which for highly eccentric binaries led to undersampling near periastron. Although a correction for undersampling was formulated (\citealt{murphyetal2016b}), it leads to decreased discriminatory power between orbits at high eccentricity \citep{murphyetal2018}. Here, we describe an approach that forward-models the time delays directly on the light curve, mitigating the sampling problem. It has the added benefit that computationally expensive series expansions in Bessel functions are no longer required.

\section{New Approach}

A major strength of the existing method is the independent calculation of $\tau_{i,j}$ for each $j$-th oscillation mode in each $i$-th segment. This allows the response of each pulsation mode to the orbit to be visualised and checked for mutual agreement (e.g. figures 1--5 of \citealt{murphyetal2014}). A weighted-average time delay is calculated across all $j$ modes in each segment $i$, with the relative mode amplitudes $A_j$ and $B_j$ acting as weights (stronger modes are weighted more heavily). In our formulation here, the mode amplitudes will also act as weights since the contribution of each $j$-th mode to the luminosity variations is scaled by the amplitude $A_j$ of that mode (Eq.\,\ref{eq:luminosity}), except that $\tau_j$ is now calculated at each individual observation $t$.

We begin by separating the time-delay equation (Eq.\,\ref{eq:delay}) into a shape and an amplitude component
\begin{eqnarray}
\tau_{t, j} = \mathcal{A}_j \psi_t,
\label{eq:shape}
\end{eqnarray}
where
\begin{eqnarray}
\psi_t = -\frac{1-e^2}{1+e \cos f} \sin(f+\varpi)
\label{eq:psi}
\end{eqnarray}
and $\mathcal{A}_j$ is evaluated for each $j$-th mode. Modes can be grouped together by their $\mathcal{A}_j$, which is particularly useful when there are two pulsators in the same binary system (e.g. figure 6 of \citealt{murphyetal2014}).

We construct the design matrix, $D_j$, for each $j$-th mode and each observation time $t_n$, and we combine these into the master design matrix $D$, which has $N$ rows and $2J$ columns:

\begin{eqnarray}
D_j = \left( \begin{array}{cc}
	\cos (\omega_j[t_1 - \tau_{1,j}])& \sin (\omega_j[t_1 - \tau_{1,j}]) \\
	\cos (\omega_j[t_2 - \tau_{2,j}])& \sin (\omega_j[t_2 - \tau_{2,j}]) \\
	\vdots & \vdots\\
	\cos (\omega_j[t_n - \tau_{n,j}])& \sin (\omega_j[t_n - \tau_{n,j}]) \\
	\vdots & \vdots \\
	\cos (\omega_j[t_N - \tau_{N,j}])& \sin (\omega_j[t_N - \tau_{N,j}])
	\end{array}\right)
\label{eq:Dj}
\end{eqnarray}
and
\begin{eqnarray}
D = (D_1~D_2~\dots~D_j~\dots~D_J).
\label{eq:D}
\end{eqnarray}

The amplitude coefficients $A_j$ and $B_j$ are collected in the column matrices $w_j$
\begin{eqnarray}
w_j = \left( \begin{array}{c}
	A_j \\
	B_j
	\end{array} \right),
\label{eq:DW}
\end{eqnarray}
which are combined into the matrix of weights, $w$
\begin{eqnarray}
w = \left( \begin{array}{c}
	w_1 \\
	w_2 \\
	\vdots \\
	w_j \\
	\vdots \\
	w_J \\
	\end{array} \right).
\label{eq:w}
\end{eqnarray}
The variance in the light curve, $y$ (Eq.\,\ref{eq:luminosity}), is then expressed as
\begin{eqnarray}
y = D \cdot w
\label{eq:DW}
\end{eqnarray}
and the maximum likelihood values for the weights can be found by linear regression
\begin{eqnarray}
\hat w = (D^T \cdot D)^{-1}(D^T \cdot y_n)
\label{eq:What}
\end{eqnarray}
where $D^T$ is the transpose of $D$ and $y_n$ is the luminosity variation at time $t_n$. The value of the likelihood at this maximum is
\begin{eqnarray}
\hat L = \sum_{n=1}^N \frac{\left( y_n - (D\cdot \hat w)_n \right)^2}{\sigma^2}
\label{eq:chisq}
\end{eqnarray}
where $\sigma$ is the standard deviation of each measurement, $y_n$. \red{SJM: this can't be the maximum likelihood, since it gets larger when one strays from $\hat w$. This is $\chi^2$, which is related to $\hat L$ in some other way, such as $\hat L = {\rm exp}[-\chi^2/2]$.}

\subsection{Calculating the uncertainties}

For a given parameter $\theta$ in the model, whose optimum value is $\hat \theta$, the log-likelihood is expressed as
\begin{eqnarray}
\log L = -\frac{1}{2}\frac{(\theta - \hat \theta)^2}{\sigma_{\theta}^2},
\end{eqnarray}
so the uncertainty $\sigma_{\theta}^2$ can be found by taking the second derivative of the log-likelihood
\begin{eqnarray}
\frac{d \log L}{d\theta} &=& -\frac{(\theta - \hat \theta)}{\sigma_{\theta}^2}\\
\frac{d^2 \log L}{d\theta^2} &=& -\frac{1}{\sigma_{\theta}^2}\\
\sigma_{\theta}^2 &=& -\frac{1}{\frac{d^2 \log L}{d\theta^2}}.
\label{eq:sigma}
\end{eqnarray}
The arbitrary number of model parameters $M$ are collected together in the Hessian, $H$,
\begin{eqnarray}
H = \left( \begin{array}{cccc}
	\sigma_{\theta_1}\sigma_{\theta_1}&~\sigma_{\theta_1}\sigma_{\theta_2}&~\dots&~\sigma_{\theta_1}\sigma_{\theta_M} \\
	\sigma_{\theta_2}\sigma_{\theta_1}&~\sigma_{\theta_2}\sigma_{\theta_2}&~\dots&~\sigma_{\theta_2}\sigma_{\theta_M} \\
	\vdots&~\vdots&~\vdots&~\vdots\\
	\sigma_{\theta_M}\sigma_{\theta_1}&~\sigma_{\theta_M}\sigma_{\theta_2}&~\dots&~\sigma_{\theta_M}\sigma_{\theta_M} \\
	\end{array}\right)
\label{eq:hessian}
\end{eqnarray}
which is a square matrix with $M$ rows and columns. \red{In addition to this, we define this beast called $\Sigma_{\theta}$, which is the negative inverse of the Hessian,}
\begin{eqnarray}
\Sigma_{\theta} = -(H)^{-1}.
\end{eqnarray}

\section{To do:}

We only produce keplerian orbits.\\
Could have spline fitted to individual $\tau(t_n)$ measurements.\\
Note integration time is assumed to be zero.\\
Adding in RV data (be careful -- use same parametrization).\\

\subsection{Examples}

Short period binaries:
\begin{enumerate}
\item KIC\,10080843 -- 15-d PB2.
\item KIC\,6780873 -- 9.15-d PB1.
\item Simulated TESS subdwarfs, possibly super-Nyquist of 2-min cadence.
\end{enumerate}
Other examples we might consider:
\begin{enumerate}
\item \textit{Kepler} $\delta$\,Sct using the Nyquist aliases.
\end{enumerate}

%\begin{figure}
%\begin{center}
%\includegraphics[width=0.48\textwidth]{example.eps}
%\caption{Example of a caption.}
%\label{fig:example}
%\end{center}
%\end{figure}
%
%\begin{table}
%\centering
%\caption{I'm a table caption.}
%\label{tab:example}
%\begin{tabular}{rl}
%\toprule
%Note & Reference\\
% & km\,s$^{-1}$\\
%\midrule
%Some & Content\\
%\bottomrule
%\end{tabular}
%\end{table}

\section*{Acknowledgements}

This research was supported by the Simons Foundation and the Australian Research Council.

%%%%%%%%%%%%%%%%%%%%%%%%%%%%%%%%%%%%%%%%%%%%%

%%%%%%%%%%%%%%%%%%%% REFERENCES %%%%%%%%%%%%%%%%%

\bibliographystyle{mnras}
\bibliography{sjm_bibliography} % name of .bib file, without extension

%%%%%%%%%%%%%%%%%%%%%%%%%%%%%%%%%%%%%%%%%%%%%%

% endmatter

\bsp	% typesetting comment
\label{lastpage}
\end{document}
